\documentclass[conference]{IEEEtran}
\IEEEoverridecommandlockouts
% The preceding line is only needed to identify funding in the first footnote. If that is unneeded, please comment it out.
\usepackage{cite}
\usepackage{amsmath,amssymb,amsfonts}
\usepackage{algorithmic}
\usepackage{graphicx}
\usepackage{textcomp}
\usepackage{xcolor}
\def\BibTeX{{\rm B\kern-.05em{\sc i\kern-.025em b}\kern-.08em
    T\kern-.1667em\lower.7ex\hbox{E}\kern-.125emX}}
\begin{document}

\title{Lidar Based SLAM Algoritm\\
}

\author{\IEEEauthorblockN{1\textsuperscript{st} Ruihua Cai}
\IEEEauthorblockA{\textit{Computer Engineering Department} \\
\textit{San Jose State University}\\
San Jose, CA, US \\
ruihua.cai@sjsu.edu}
\and
\IEEEauthorblockN{2\textsuperscript{nd} Ziyuan Yan}
\IEEEauthorblockA{\textit{Computer Engineering Department} \\
\textit{San Jose State University}\\
San Jose, CA, US \\
ruihua.cai@sjsu.edu}
\and
\IEEEauthorblockN{3\textsuperscript{rd} Ruixiang Huang}
\IEEEauthorblockA{\textit{Computer Engineering Department} \\
\textit{San Jose State University}\\
San Jose, CA, US \\
ruihua.cai@sjsu.edu}
}

\maketitle

\begin{abstract}
In the field of mobile robotics, localization is a very important issue. One particularly difficult problem is how to enable a robot to determine its position in an unknown environment. Simultaneous localization and mapping (SLAM) is one way to solve this problem. Also, Lidar is the most common sensor used by mobile robots to implement SLAM. This project is intended to explore the difficulties of SLAM algorithms by implementing some simple 2D Lidar-based SLAM algorithms. This project will be simulated in Matlab to determine the accuracy and efficiency of the algorithm.
\end{abstract}

\begin{IEEEkeywords}
Lidar, SLAM, ICP
\end{IEEEkeywords}

\section{Introduction}
In real-world robot research, Simultaneous localization and mapping (SLAM) is a critical topic for helping robots to understand their location in a space and create a map for collecting and analyzing the surrounding environment. 

There are serval ways to implement SLAM on robots. Such as camera, sonar, and Lidar. Using the camera, we could use the image processing algorithm to get a graph of the current environment. Even though we could get a high-resolution image from it, due to the sensor itself doesn't provide very accurate distance data, and noise under too bright or dark conditions could affect the result. Camera-based SLAM is not common for people to research. Sonar-based SLAM is the majority applied to underwater robot SLAM. Due to the environmental issue, it could not rely on light-based sensors to perform SLAM. The best choice for underwater robots to do SLAM is based on sonar, which can handle the undercurrent in water\cite{b4}. Using Lidar is the most popular solution for solving SLAM issues. It could provide a wide range of searching angle and distance information, which is excellent for developing SLAM research. 

Since Lidar is the most common and affordable device for SLAM research. By collecting the surrounding Lidar information, we could get the outline of the surrounding area. Due to the advantages of Lidar, which include the distance information, we could determine the distance between the robot and the wall or obstacles around the robot. In this project, we will develop two kinds of SLAM algorithms without the data of odometry. The first algorithm is based on the prediction robot’s movement to match the similarity between the previous scan and the current scan result. If the robot could match the pattern of the scan result, we can say the robot moved in this direction and distance. The second algorithm is based on the Iterative Closest Point (ICP) algorithm. By comparing the closest point shifting between the previous scan and the current scan, we could get a vector shifting that is suitable for most of the points, and this vector is the movement of the robot. 

\section{Related Work and Background Material}
By studying Wolfgang \& Damon\cite{b1} and John \& John\cite{b4}'s article, we have a more comprehensive understanding of the different sensor-based robot SLAM algorithms and how it benefits human life. This involved our interest in researching different kinds of SLAM approaches.  In Shan's article \cite{b2}, he demonstrates a way to implement SLAM by combining multiple sensor data and using the Extended Karmal Filter (EKF) to perform sensor fusion for calculating SLAM. We study how this author uses 2D-Lidar to collect the surrounding environment data and understand the basic theory of localization. Finally, from Shaofeng \& Jingyu's article \cite{b3}, we found another approach to implementing robot SLAM based on Lidar. Through the above research and study, these articles gave us basic knowledge and direction about how to implement SLAM into our project.

\section{Technical  approach}
\subsection{Pattern Match SLAM}
In order to do the SLAM computation, transfering Lidar data into Cartesian coordinate and mapping into a customized resoultion output is easier computation. After getting the second set of data. By looping all the direction of to find out the difference between curren scan vs previous scan to determine which direction and how long distance did the robot move.

To collect a comperhensive mapping data, the robot need to finish a loop running in the area. By cumulating the data, finally we could get the map data of the current space.
\subsection{Iterative Closest Point SLAM}
The IEEEtran class file is used to format your paper and style the text. All margins, 
column widths, line spaces, and text fonts are prescribed; please do not 
alter them. You may note peculiarities. For example, the head margin
measures proportionately more than is customary. This measurement 
and others are deliberate, using specifications that anticipate your paper 
as one part of the entire proceedings, and not as an independent document. 
Please do not revise any of the current designations.

\section{Results Analyze}
Before you begin to format your paper, first write and save the content as a 
separate text file. Complete all content and organizational editing before 
formatting. Please note sections below for more information on 
proofreading, spelling and grammar.

Keep your text and graphic files separate until after the text has been 
formatted and styled. Do not number text heads---{\LaTeX} will do that 
for you.

\subsection{Abbreviations and Acronyms}\label{AA}
Define abbreviations and acronyms the first time they are used in the text, 
even after they have been defined in the abstract. Abbreviations such as 
IEEE, SI, MKS, CGS, ac, dc, and rms do not have to be defined. Do not use 
abbreviations in the title or heads unless they are unavoidable.

\subsection{Equations}
Number equations consecutively. To make your 
equations more compact, you may use the solidus (~/~), the exp function, or 
appropriate exponents. Italicize Roman symbols for quantities and variables, 
but not Greek symbols. Use a long dash rather than a hyphen for a minus 
sign. Punctuate equations with commas or periods when they are part of a 
sentence, as in:
\begin{equation}
a+b=\gamma\label{eq}
\end{equation}

Be sure that the 
symbols in your equation have been defined before or immediately following 
the equation. Use ``\eqref{eq}'', not ``Eq.~\eqref{eq}'' or ``equation \eqref{eq}'', except at 
the beginning of a sentence: ``Equation \eqref{eq} is . . .''

\subsection{Figures and Tables}
\paragraph{Positioning Figures and Tables} Place figures and tables at the top and 
bottom of columns. Avoid placing them in the middle of columns. Large 
figures and tables may span across both columns. Figure captions should be 
below the figures; table heads should appear above the tables. Insert 
figures and tables after they are cited in the text. Use the abbreviation 
``Fig.~\ref{fig}'', even at the beginning of a sentence.

\begin{table}[htbp]
\caption{Table Type Styles}
\begin{center}
\begin{tabular}{|c|c|c|c|}
\hline
\textbf{Table}&\multicolumn{3}{|c|}{\textbf{Table Column Head}} \\
\cline{2-4} 
\textbf{Head} & \textbf{\textit{Table column subhead}}& \textbf{\textit{Subhead}}& \textbf{\textit{Subhead}} \\
\hline
copy& More table copy$^{\mathrm{a}}$& &  \\
\hline
\multicolumn{4}{l}{$^{\mathrm{a}}$Sample of a Table footnote.}
\end{tabular}
\label{tab1}
\end{center}
\end{table}

\begin{figure}[htbp]
\centerline{\includegraphics{fig1.png}}
\caption{Example of a figure caption.}
\label{fig}
\end{figure}

Figure Labels: Use 8 point Times New Roman for Figure labels. Use words 
rather than symbols or abbreviations when writing Figure axis labels to 
avoid confusing the reader. As an example, write the quantity 
``Magnetization'', or ``Magnetization, M'', not just ``M''. If including 
units in the label, present them within parentheses. Do not label axes only 
with units. In the example, write ``Magnetization (A/m)'' or ``Magnetization 
\{A[m(1)]\}'', not just ``A/m''. Do not label axes with a ratio of 
quantities and units. For example, write ``Temperature (K)'', not 
``Temperature/K''.

\section{Conclusion}
Before you begin to format your paper, first write and save the content as a 
separate text file. Complete all content and organizational editing before 
formatting. Please note sections below for more information on 
proofreading, spelling and grammar.

Keep your text and graphic files separate until after the text has been 
formatted and styled. Do not number text heads---{\LaTeX} will do that 
for you.

\begin{thebibliography}{00}
\bibitem{b1} W. Hess, D. Kohler, H. Rapp and D. Andor, "Real-time loop closure in 2D LIDAR SLAM," 2016 IEEE International Conference on Robotics and Automation (ICRA), 2016, pp. 1271-1278, doi: 10.1109/ICRA.2016.7487258.
\bibitem{b2} S. Huang, H.-Z. Huang, Q. Zeng, and P. Huang, "A Robust 2D Lidar SLAM Method in Complex Environment", Photonic Sensors, vol. 12, no. 4, p. 220416, Apr. 2022.
\bibitem{b3} S. Wu and J. Lin, "A novel SLAM framework based on 2D LIDAR", Journal of Physics: Conference Series, vol. 1650, no. 2, p. 22104, 2020.
\bibitem{b4} J. Folkesson and J. Leonard, "Autonomy through SLAM for an underwater robot", in Robotics Research, Springer, 2011, pp. 55-70.
\end{thebibliography}
\vspace{12pt}

\end{document}
